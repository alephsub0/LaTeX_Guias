\documentclass[a4,10pt]{aleph-notas}

%% --> Paquetes adicionales
\usepackage{multicol}
\usepackage{marginnote}
\usepackage{metalogo}
%% --> Paquetes comunes
\usepackage{listings}
\usepackage{enumitem}
\usepackage{lipsum}

%% --> Definición de colores
\definecolor{codegreen}{HTML}{A5BE00}
\definecolor{codegray}{rgb}{0.5,0.5,0.5}
\definecolor{codepurple}{rgb}{0.58,0,0.82}
\definecolor{backcolour}{rgb}{0.95,0.95,0.92}

%% --> Estilo para código
\lstdefinestyle{mystyle}{
    language={[LaTeX]TeX}, % lenguaje
    basicstyle=\bfseries\ttfamily,
    keywordstyle=\color{colordef},
    commentstyle=\color{codegreen},
    backgroundcolor=\color{gray!15},
    showstringspaces=false,
    flexiblecolumns=true,
    stringstyle=\ttfamily\color{blue},
    extendedchars=true,
    emph={rm,bf,it,sf}, %...
    literate=%
    *{$}{{{\color{red}\$}}}1 % produce $ en rojo
    {$$}{{{\color{red}\$\$}}}1
    {ó}{{\'o}}1%
    {í}{{\'i}}1%
    {á}{{\'a}}1%
    {ú}{{\'u}}1%
}

%% --> Selección de estilo para el código
\lstset{
    style=mystyle,escapeinside={(*@}{@*)}
}

% Blancos tipográficos
\newcommand{\mq}{\hspace{0.5em}}  %medio cuadratín
\newcommand{\tq}{\hspace{0.33em}} % un terio de cuadratín
\newcommand{\qq}{\hspace{0.25em}} % un cuarto de cuadratín
\newcommand{\fs}{\hspace{0.125em}} % un octavo de cuadratín
\newcommand{\ep}{\hspace{0.05em}} % espacio de pelo

%% --> Nota para el material
\newcommand{\informacion}{\noindent\footnotesize{\color{colordef}
El presente material fue desarrollado por:

\noindent
\textbf{Daniel Lara}

\emph{Facultad de Ciencias, Escuela Politécnica Nacional}

\noindent
\textbf{Andrés Merino}

\emph{Facultad de Ciencias Exactas y Naturales, Pontificia Universidad Católica del Ecuador}


\medskip\noindent
La versión actual del material es 1.2-(Mayo 2021). En caso de encontrar inconsistencias o errores en el presente material se pueden comunicar a \href{mailto:daniel.lara@alephsub0.org}{daniel.lara@alephsub0.org}. Para más información puedes visitar nuestro sitio web: \href{https://alephsub0.org}{alephsub0.org}

\medskip\noindent
\includegraphics[height=12pt]{Imagenes/cc.xlarge.png} \includegraphics[height=12pt]{Imagenes/by.xlarge.png} \includegraphics[height=12pt]{Imagenes/nc.xlarge.png} \begin{minipage}[c]{0.85\textwidth}Esta obra se encuentra bajo licencia Atribución-NoComercial-CompartirIgual 4.0 Internacional (CC BY-NC-SA 4.0) Para más información puede visitar: \url{https://creativecommons.org/licenses/by-nc-sa/4.0/}\end{minipage}

\medskip\noindent
Si deseas colaborar con el desarrollo de este material, el código fuente está disponible en:   
\url{https://github.com/alephsub0/LaTeX_Guias.git}. Cualquier aporte (\emph{Pull request}) será de gran ayuda para mejorar este material. 

%% -- > Aquí se incluyen los nombres de los colaborades de estas guias:
\medskip\noindent
Agradecimientos: Katheryn Yánes
}}

%%--> Formato para títulos
\titleformat{name=\section,numberless}[display]
  {\vspace*{-2mm}\bfseries\scshape\centering}
    {}{1ex}
    {\color{colortext}\large\titlerule\vspace{.05ex}
     }
    [\color{colortext}\vspace{.2ex}\titlerule]

\titleformat{\subsubsection}
    {\color{colortext}\normalsize\bfseries}
    {\thesubsubsection}{1em}{}
    
%% --> Datos de las guias
\universidad{Curso de \LaTeX}
\autor{Proyecto Alephsub0}
\materia{Introducción a \LaTeX}

%% --> Logos de las guias
\logouno[4.5cm]{Imagenes/Logos/LogoAlephsub0-02.png}
\longtitulo{0.6\linewidth}
\fecha{Abril de 2021}

% -- Datos del libro
\nota{Guía 09}
\tema{Fuentes}

%%--> Opciones adicionales

%%%%%%%%%%%%%%%%%%%%%%%%%%%%%%%%%%%%%%%%
%%%%%%%%%% Comienzo del documento
%%%%%%%%%%%%%%%%%%%%%%%%%%%%%%%%%%%%%%%%

\begin{document}

\encabezado

\informacion


\tableofcontents


\section{Notas adicionales sobre Fuentes}

\subsection{Más tamaños de fuentes}

En ocasiones resulta útil modificar el tamaño de las fuentes más allá de las opciones que \LaTeX{} nos presenta por defecto. Para ello, vamos a emplear el paquete \verb@\usepackage{anyfontsize}@. Este paquete 
\begin{lstlisting}[frame=single]
    {\fontsize{18pt}{21.6pt}\selectfont Texto 1 : Aqui puedes colocar algun texto. 
    Sin embargo, necesitas un texto suficientemente largo para notar la diferencia.
    \par}

    \bigskip
    
    {\fontsize{22pt}{26.4pt}\selectfont Texto 2 :Aqui puedes colocar algun texto. 
    Sin embargo, necesitas un texto suficientemente largo para notar la diferencia. 
    \par}
\end{lstlisting}

{\fontsize{18pt}{21.6pt}\selectfont Texto 1 : Aquí puedes colocar algún texto. Sin embargo, necesitas un texto suficientemente largo para notar la diferencia.\par}

\bigskip

{\fontsize{22pt}{26.4pt}\selectfont Texto 2 :Aquí puedes colocar algún texto. Sin embargo, necesitas un texto suficientemente largo para notar la diferencia. \par}

\subsection{Codificación de fuentes}

Recordemos que los símbolos empleados en la escritura están codificados por una serie de códigos que definen cada uno de ellos, todo esto a manera de un apodo para poder identificarlos. Sin embargo y dependiendo de la codificación que estemos usando podemos incluir cada uno de estos caracteres con el siguiente comando.

\begin{lstlisting}[frame=single]
    \char60
\end{lstlisting}

\begin{center}
{ \fboxsep 12pt
\fcolorbox {black}{white}{
\begin{minipage}[t]{0.1\textwidth}
\centering
\char60
\end{minipage}
} }
\end{center}

\subsection{Familias de fuentes}

\begin{table}[H]
    \centering
    \begin{tabular}{cl}
        \hline
        cmr & \fontfamily{cmr}\selectfont Computer Modern Roman (predeterminada) \\
        cmss & \fontfamily{cmss}\selectfont Computer Modern Sans\\
        cmtt & \fontfamily{cmtt}\selectfont Computer Modern Typewriter\\
        cmm & \fontfamily{cmm}\selectfont Computer Modern Math Italic\\
        cmsy & \fontfamily{cmsy}\selectfont Computer Modern Math Symbols\\
        \hline
    \end{tabular}
\end{table}

\begin{table}[H]
    \centering
    \begin{tabular}{cl}
        \hline
        cmex & \fontfamily{cmex}\selectfont Computer Modern Math Extensions\\
        ptm &\fontfamily{ptm}\selectfont Adobe Times \\
        phv & \fontfamily{phv}\selectfont Adobe Helvetica\\
        pcr & \fontfamily{pcr}\selectfont Adobe Courier\\
        \hline
    \end{tabular}
    \caption{Tabla de familia de fuentes}
\end{table}

\subsection{Peso de la fuente}

\begin{table}[H]
    \centering
    \begin{tabular}{cl}
        \hline
        m & \fontseries{m}\selectfont Medium\\
        b &\fontseries{b}\selectfont Bold \\
        bx & \fontseries{bx}\selectfont Bold extended\\
        sb & \fontseries{sb}\selectfont Semi bold\\
        c & \fontseries{c}\selectfont condensed \\
        \hline
    \end{tabular}
    \caption{Tabla de peso de las fuentes}
\end{table}

\subsection{Forma de la fuente}

\begin{table}[H]
    \centering
    \begin{tabular}{cl}
        \hline
        n & \fontshape{n}\selectfont Normal\\
        it &\fontshape{it}\selectfont Italic \\
        sl & \fontshape{sl}\selectfont Slanted (``oblicua'')\\
        sc & \fontshape{sc}\selectfont Caps and small caps\\
        \hline
    \end{tabular}
    \caption{Tabla de forma de las fuentes}
\end{table}

\section{Nuevas fuentes}

En esta sección daremos una breve introducción al uso de fuentes nativas del computador. Para su funcionamiento, \LaTeX{} usa las fuentes disponibles en la instalación del compilador, es decir, no accede directamente a las fuentes que se encuentran instaladas en el sistema operativo. Para obtener acceso a las mismas y poder trabajar con ellas es necesario usar un compilador especial, aquí vamos a introducir la compilación con \XeLaTeX{}. Este nuevo motor de compilación introduce el soporte a fuentes TrueType y OpenType, en este curso no se profundiza más sobre este tema pero basta con saber que se refiere a «formatos» de fuentes. Una vez seleccionado el nuevo compilador y verificando la disponibilidad de la fuente, es necesario cargarla en el documento, para ello, primero, cargamos el paquete que nos permite gestionar las nuevas fuentes:

\begin{lstlisting}[frame=single]
\usepackage{fontspec}
\end{lstlisting}

Cuando hacemos uso de \XeLaTeX la codificación por defecto soporta el estándar \texttt{UTF-8}. Luego de ello, declaramos la fuente a usar

\begin{lstlisting}[frame=single]
\setmainfont{Times New Roman}
\end{lstlisting}

Para el ejemplo declaramos que la nueva fuente a usar es Times New Roman, esta va a reemplazar a Computer Modern que es la familia por defecto en \LaTeX{}. En algunos casos podemos desear escoger fuentes personalizadas para cada estilo disponible, en este caso podemos emplear el siguiente código como ejemplo:

\begin{lstlisting}[frame=single]
\setromanfont{Times New Roman}
\setsansfont{Arial}
\setmonofont[Color={0019D4}]{Courier New}
\end{lstlisting}

La lista completa de fuentes disponibles en Overleaf se puede encontrar en el siguiente enlace: \href{https://es.overleaf.com/learn/latex/Questions/Which\%20OTF\%20or\%20TTF\%20fonts\%20are\%20supported\%20via\%20fontspec\%3F}{Listado de fuentes}. Por otra parte, para descagar nuevas fuentes uno de los repositorios más populares es el proporcionado por Google: \url{https://fonts.google.com/}.


Finalmente, en casos particulares podemos requerir usar una fuente no instalada en el compilador, en estos casos debemos disponer de todos los archivos correspondientes a la familia de la fuente a usar. Cargamos los archivos en el directorio y los declaramos. En el siguiente ejemplo se dispone de los siguientes archivos:

\begin{itemize}
    \item \texttt{BodoniB.ttf}
    \item \texttt{BodoniI.ttf}
    \item \texttt{BodoniR.ttf}
\end{itemize}

Estos archivos corresponden a las diferentes variaciones de la fuente y se las declara con el siguiente código:

\begin{lstlisting}[frame=single]
\setromanfont[
BoldFont=BodoniB.ttf,
ItalicFont=BodoniI.ttf,
]{BodoniR.ttf}
\end{lstlisting}

Y, podemos comprobar el resultado con el siguiente código:

\begin{lstlisting}[frame=single]
Este es un texto en la familia regular de Bodoni MT.

{\it Este es un texto en la familia regular de Bodoni MT.}

{\bf Este es un texto en la familia regular de Bodoni MT.}

\end{lstlisting}

\end{document} 