\documentclass[a4,10pt]{aleph-notas}

%% --> Comandos adicionales
\usepackage{textcomp}
\usepackage{multirow}
\usepackage{booktabs}
\usepackage{longtable}
\usepackage{colortbl}
\usepackage{csquotes}
\usepackage{float}
%% --> Paquetes comunes
\usepackage{listings}
\usepackage{enumitem}
\usepackage{lipsum}

%% --> Definición de colores
\definecolor{codegreen}{HTML}{A5BE00}
\definecolor{codegray}{rgb}{0.5,0.5,0.5}
\definecolor{codepurple}{rgb}{0.58,0,0.82}
\definecolor{backcolour}{rgb}{0.95,0.95,0.92}

%% --> Estilo para código
\lstdefinestyle{mystyle}{
    language={[LaTeX]TeX}, % lenguaje
    basicstyle=\bfseries\ttfamily,
    keywordstyle=\color{colordef},
    commentstyle=\color{codegreen},
    backgroundcolor=\color{gray!15},
    showstringspaces=false,
    flexiblecolumns=true,
    stringstyle=\ttfamily\color{blue},
    extendedchars=true,
    emph={rm,bf,it,sf}, %...
    literate=%
    *{$}{{{\color{red}\$}}}1 % produce $ en rojo
    {$$}{{{\color{red}\$\$}}}1
    {ó}{{\'o}}1%
    {í}{{\'i}}1%
    {á}{{\'a}}1%
    {ú}{{\'u}}1%
}

%% --> Selección de estilo para el código
\lstset{
    style=mystyle,escapeinside={(*@}{@*)}
}

% Blancos tipográficos
\newcommand{\mq}{\hspace{0.5em}}  %medio cuadratín
\newcommand{\tq}{\hspace{0.33em}} % un terio de cuadratín
\newcommand{\qq}{\hspace{0.25em}} % un cuarto de cuadratín
\newcommand{\fs}{\hspace{0.125em}} % un octavo de cuadratín
\newcommand{\ep}{\hspace{0.05em}} % espacio de pelo

%% --> Nota para el material
\newcommand{\informacion}{\noindent\footnotesize{\color{colordef}
El presente material fue desarrollado por:

\noindent
\textbf{Daniel Lara}

\emph{Facultad de Ciencias, Escuela Politécnica Nacional}

\noindent
\textbf{Andrés Merino}

\emph{Facultad de Ciencias Exactas y Naturales, Pontificia Universidad Católica del Ecuador}


\medskip\noindent
La versión actual del material es 1.2-(Mayo 2021). En caso de encontrar inconsistencias o errores en el presente material se pueden comunicar a \href{mailto:daniel.lara@alephsub0.org}{daniel.lara@alephsub0.org}. Para más información puedes visitar nuestro sitio web: \href{https://alephsub0.org}{alephsub0.org}

\medskip\noindent
\includegraphics[height=12pt]{Imagenes/cc.xlarge.png} \includegraphics[height=12pt]{Imagenes/by.xlarge.png} \includegraphics[height=12pt]{Imagenes/nc.xlarge.png} \begin{minipage}[c]{0.85\textwidth}Esta obra se encuentra bajo licencia Atribución-NoComercial-CompartirIgual 4.0 Internacional (CC BY-NC-SA 4.0) Para más información puede visitar: \url{https://creativecommons.org/licenses/by-nc-sa/4.0/}\end{minipage}

\medskip\noindent
Si deseas colaborar con el desarrollo de este material, el código fuente está disponible en:   
\url{https://github.com/alephsub0/LaTeX_Guias.git}. Cualquier aporte (\emph{Pull request}) será de gran ayuda para mejorar este material. 

%% -- > Aquí se incluyen los nombres de los colaborades de estas guias:
\medskip\noindent
Agradecimientos: Katheryn Yánes
}}

%%--> Formato para títulos
\titleformat{name=\section,numberless}[display]
  {\vspace*{-2mm}\bfseries\scshape\centering}
    {}{1ex}
    {\color{colortext}\large\titlerule\vspace{.05ex}
     }
    [\color{colortext}\vspace{.2ex}\titlerule]

\titleformat{\subsubsection}
    {\color{colortext}\normalsize\bfseries}
    {\thesubsubsection}{1em}{}
    
%% --> Datos de las guias
\universidad{Curso de \LaTeX}
\autor{Proyecto Alephsub0}
\materia{Introducción a \LaTeX}

%% --> Logos de las guias
\logouno[4.5cm]{Imagenes/Logos/LogoAlephsub0-02.png}
\longtitulo{0.6\linewidth}
\fecha{Abril de 2021}


% -- Datos del libro
\nota{Guía 10}
\tema{Tablas}

%% ---> Nuevos tipos de columnas
\newcolumntype{G}{>{\color{red}$\pm$}c}
\newcolumntype{C}[1]{>{\hspace{0pt}\centering\arraybackslash}m{#1}}
\newcolumntype{L}[1]{>{\raggedright\arraybackslash}m{#1}}

\titleformat{name=\section,numberless}[display]
  {\vspace*{-2mm}\bfseries\scshape\centering}
    {}{1ex}
    {\color{colortext}\large\titlerule\vspace{.05ex}
     }
    [\color{colortext}\vspace{.2ex}\titlerule]

\definecolor{micolor}{rgb}{.6,.5,.4}

%%%%%%%%%%%%%%%%%%%%%%%%%%%%%%%%%%%%%%%%
%%%%%%%%%% Comienzo del documento
%%%%%%%%%%%%%%%%%%%%%%%%%%%%%%%%%%%%%%%%

\begin{document}

\encabezado

\informacion

\tableofcontents

\section{Tablas}

Para la creación de tablas usamos el ambiente \verb@tabular@, este tiene la siguiente estructura:

\begin{lstlisting}[frame=single]
\begin{tabular}{(*@\emph{formato}@*)}
(*@\emph{contenido de la tabla}@*)
\end{tabular}
\end{lstlisting}

El parámetro \emph{formato} especifica la alineación de la columna y los separadores que existen en ella; a continuación se encuentra una lista de los formatos de columna disponibles. Para la creación de una tabla es necesario especificar el número de columnas, mientras que no se requiere indicar el número de filas.

\begin{itemize}
    \item \verb@l@ alinea la columna a la izquierda;
    \item \verb@r@ alinea la columna a la derecha;
    \item \verb@c@ alinea la columna al centro;
    \item \verb@p{@\emph{longitud}\verb@}@ indica una columna con la longitud especificada, las unidades de medida disponibles son las especificadas en la guía 1.
    \item \verb@|@ indica una linea vertical de separación entre columnas.
\end{itemize}

Por otra parte, dentro del ambiente se usan los siguientes delimitadores:

\begin{itemize}
    \item \verb@&@ indica un cambio de columna;
    \item \verb@\\@ indica un salto de fila;
    \item \verb@\hline@ inserta una linea horizontal en la tabla;
    \item \verb@\cline{a-b}@ inserta una linea horizontal desde la columna $a$ a la columna $b$.
\end{itemize}

Un ejemplo para la generación de una tabla se encuentra a continuación:

\begin{lstlisting}[frame=single]
\begin{tabular}{|c|c|c|}
    \hline
    \textbf{Ciclo N$^{\circ}$} &	\textbf{Temperatura} & \textbf{Tiempo} \\ \hline
    1   & $245 \pm 5,5$ &	3 \\ \hline
    2	& $260 \pm 5,5$ &	8 \\ \hline
    3	& $275 \pm 5,5$ &	8 \\ \hline
    4	& $287 \pm 5,5$ &	8 \\ \hline
\end{tabular}
\end{lstlisting}

\begin{center}
  \begin{tabular}{|c|c|c|}
   \hline
    \textbf{Ciclo N$^{\circ}$} &	\textbf{Temperatura} & \textbf{Tiempo} \\ \hline
    1   & $245 \pm 5,5$ &	3 \\ \hline
    2	& $260 \pm 5,5$ &	8 \\ \hline
    3	& $275 \pm 5,5$ &	8 \\ \hline
    4	& $287 \pm 5,5$ &	8 \\ \hline
  \end{tabular}
\end{center}

Notemos que no es necesario incluir lineas verticales que separan a las columnas:

\begin{lstlisting}[frame=single]
\begin{tabular}{ccc}
    \toprule
    \textbf{Ciclo N$^\circ$} &	\textbf{Temperatura} & \textbf{Tiempo} \\ \midrule
    1   & $245 \pm 5,5$ &	3 \\[1mm]
    2	& $260 \pm 5,5$ &	8 \\[1mm]
    3	& $275 \pm 5,5$ &	8 \\[1mm]
    4	& $287 \pm 5,5$ &	8 \\
    \bottomrule
\end{tabular}
\end{lstlisting}


\begin{center}
  \begin{tabular}{|cc|c|}
   \hline
    \textbf{Ciclo N$^\circ$} &	\textbf{Temperatura} & \textbf{Tiempo} \\ \hline
    1   & $245 \pm 5,5$ &	3 \\
    2	& $260 \pm 5,5$ &	8 \\
    % 3	& $275 \pm 5,5$ &	8 \\[1mm]
    % 4	& $287 \pm 5,5$ &	8 \\
    \hline
  \end{tabular}
\end{center}

\subsection{Unir celdas}

Una de las primeras operaciones con las que nos podemos topar en la generación de una tabla es la necesidad de unir ciertas celdas. Para ello vamos a recurir al comando \verb@\multicolumn@, con la siguiente sintaxis

\begin{lstlisting}[frame=single]
\multicolumn{numero de columnas}{alineacion}{texto a colocar}
\end{lstlisting}

\subsection{Paquete \emph{booktabs}}

Este paquete se utiliza para tener algunas opciones adicionales para tablas como lineas con un grosor distinto para una mejor presentación.


\begin{lstlisting}[frame=single]
\begin{tabular}{cccccccc}
    \toprule
    \multirow{2}{*}{\textbf{COMPOSICION}} & \multicolumn{7}{c}{\textbf{FORMULACION}}\\ 
    \cmidrule{2-8}
    & \textbf{43} & \textbf{44} & \textbf{45} & \textbf{46} & \textbf{47} & 
    \textbf{48} & \textbf{49}\\ \midrule
    Poliamida-colageno    & 34,8  & 34,4 &	35  & 25  &	40  & 34,9  & 34,95\\[1mm] 
    Acrilamida            & 45	   & 45	  & 45	& 45  & 45	& 45	& 45   \\[1mm]
    Etilenglicol          & 10    & 10	  & 10	& 10  & 10	& 10	& 10   \\[1mm]
    Reticulante (DMEGL)   & 0,2   & 0,6  &--- & --- &	--- & 0,10	& 0,05 \\[1mm]
    Colageno hidrolizado  & 10	   & 10	  & 10	& 20  &5	& 10	& 10   \\[1mm]
    \bottomrule
\end{tabular}
\end{lstlisting}

\begin{center}
 \begin{tabular}{cccccccc}
  \toprule
  \multirow{2}{*}{\textbf{COMPOSICIÓN}} & \multicolumn{7}{c}{\textbf{FORMULACIÓN}}\\ 
  \cmidrule{2-8}
    & \textbf{43} & \textbf{44} & \textbf{45} & \textbf{46} & \textbf{47} & \textbf{48} & \textbf{49}\\ \midrule
  Poliamida-colágeno    & 34,8  & 34,4 &	35  & 25  &	40  & 34,9  & 34,95\\[1mm] 
  Acrilamida           & 45	   & 45	  & 45	& 45  & 45	& 45	& 45   \\[1mm] \bottomrule
%   Etilenglicol          & 10    & 10	  & 10	& 10  & 10	& 10	& 10   \\[1mm]
%   Reticulante (DMEGL)   & 0,2   & 0,6  &	--- & --- &	--- & 0,10	& 0,05 \\[1mm]
%   Colágeno hidrolizado  & 10	   & 10	  & 10	& 20  &5	& 10	& 10   \\[1mm]
 \end{tabular}
\end{center}

\subsection{Paquete \emph{tabularx}}

Muchos de los ambientes y herramientas de \LaTeX{} se ven limitados por sus caracteristicas. En el caso de las tablas, para agregar más funciones vamos a incluir el paquete \verb@tabularx@

\subsubsection{Nuevos tipos de columnas}

Una de las principales ventajas del paquete \verb@tabularx@ es la capacidad de crear nuevos tipos de columnas. A continuación veamos un ejemplo de un nuevo formato de columna para texto matemático centrado

\begin{lstlisting}[frame=single]
\newcolumntype{D}{>{$\displaystyle}c<{$}}
\newcolumntype{G}{>{\color{red}$\pm$}c}
\end{lstlisting}

\begin{lstlisting}[frame=single]
\begin{tabular}{>{\centering}m{3.5cm}cG}
   \toprule
    \textbf{Ciclo de precipitacion N$^\circ$} &	\textbf{Temperatura} 
    & \textbf{Tiempo} \\ \midrule
    1   & $245 \pm 5,5$ &	3 \\[1mm]
    2	& $260 \pm 5,5$ &	8 \\[1mm]
    3	& $275 \pm 5,5$ &	8 \\[1mm]
    4	& $287 \pm 5,5$ &	8 \\
    \bottomrule
\end{tabular}
\end{lstlisting}

\begin{center}
\begin{tabular}{>{\centering}m{3.5cm}cG}
    \toprule
    \textbf{Ciclo de precipitacion N$^\circ$} &	\textbf{Temperatura} 
    & \textbf{Tiempo} \\ \midrule
    1   & $245 \pm 5,5$ &	3 \\[1mm]
    2	& $260 \pm 5,5$ &	8 \\[1mm]
    3	& $275 \pm 5,5$ &	8 \\[1mm]
    4	& $287 \pm 5,5$ &	8 \\
    \bottomrule
\end{tabular}
\end{center}


\vspace{2cm}


\begin{lstlisting}[frame=single]
\begin{tabular}{|>{\columncolor{cyan}}c|c|>{\color{magenta}}c|}
    \hline
    \rowcolor{brown}  \textbf{Ciclo N$^\circ$} &
    \textbf{Temperatura} & \textbf{Tiempo} \\ \hline
    1   & $245 \pm 5,5$ &	3 \\ \hline
    2	& $260 \pm 5,5$ &	\cellcolor{yellow} 8 \\ \hline
    3	& $275 \pm 5,5$ &	8 \\ \hline
    4	& $287 \pm 5,5$ &	8 \\ \hline
\end{tabular}
\end{lstlisting}

\begin{center}
  \begin{tabular}{|>{\columncolor{cyan}}c|c|>{\color{magenta}}c|}
   \hline
  \textbf{Ciclo N$^\circ$} &	\textbf{Temperatura} & \textbf{Tiempo} \\ \hline
    1   & \cellcolor{yellow}$245 \pm 5,5$ &	3 \\ \hline
    {\color{red}2}	& $260 \pm 5,5$ & 8 \\ \hline
  \end{tabular}
\end{center}

\vspace{12pt}

\subsection{Tablas largas (Paquete \emph{longtable})}

Para la creación de este tipo de tablas, necesitamos del paquete \verb@longtable@ y maneja una estructura y sintaxis similar a la de una tabla usual. Sin embargo, este paquete posee una definición más minuciosa de su estructura.

\begin{itemize}
    \item 
        \verb@\endfirsthead@ Define el encabezado de la página principal
    \item
        \verb@\endhead@
            Define el encabezado que tendrá la tabla de las siguientes páginas
    \item
        \verb@\endfoot@
            Define el pie que tendrá la tabla en todas las páginas excepto en la última
    \item
        \verb@\endlastfoot@
            Define el pie que tendrá la tabla en la última página
\end{itemize}

\begin{lstlisting}[frame=single]
\begin{center}
\begin{longtable}{lll}
    \caption{Lista de Estudiantes}\\
        \toprule
        \textbf{Nombre}  & \textbf{Carrera}  &  \textbf{Correo Electronico} \\
        \midrule
    \endfirsthead
        \multicolumn{3}{l}{\footnotesize Viene de la pagina anterior}\\
        \toprule
        \textbf{Nombre}  & \textbf{Carrera}  &  \textbf{Correo Electronico} \\ \midrule
    \endhead
        \bottomrule  \multicolumn{3}{r}{\footnotesize Continua en la siguiente pagina}
    \endfoot 
        \bottomrule
    \endlastfoot
%  
    Daniel Lara      &	Matematica	& \url{daniel.lara@alephsub0.com}  \\
    Daniel Lara      &	Matematica	& \url{daniel.lara@alephsub0.com}  \\

  
    Daniel Lara      &	Matematica	& \url{daniel.lara@alephsub0.com}  \\
\end{longtable}
\end{center}
\end{lstlisting}


\begin{center}
 \begin{longtable}{lll}
  \caption{Lista de Estudiantes}\\
        \toprule
        \textbf{Nombre}  & \textbf{Carrera}  &  \textbf{Correo Electrónico} \\
        \midrule
  \endfirsthead
        \multicolumn{3}{l}{\footnotesize Viene de la página anterior}\\
        \toprule
        \textbf{Nombre}  & \textbf{Carrera}  &  \textbf{Correo Electrónico} \\ \midrule
  \endhead
        \bottomrule  \multicolumn{3}{r}{\footnotesize Continua en la siguiente página}
  \endfoot 
        \bottomrule
  \endlastfoot
%  
  Daniel Lara      &	Matemática	& \url{daniel.lara@alephsub0.com}  \\
  Daniel Lara      &	Matemática	& \url{daniel.lara@alephsub0.com}  \\
  Daniel Lara      &	Matemática	& \url{daniel.lara@alephsub0.com}  \\
  Daniel Lara      &	Matemática	& \url{daniel.lara@alephsub0.com}  \\
  Daniel Lara      &	Matemática	& \url{daniel.lara@alephsub0.com}  \\
  Daniel Lara      &	Matemática	& \url{daniel.lara@alephsub0.com}  \\
  Daniel Lara      &	Matemática	& \url{daniel.lara@alephsub0.com}  \\
  Daniel Lara      &	Matemática	& \url{daniel.lara@alephsub0.com}  \\
  Daniel Lara      &	Matemática	& \url{daniel.lara@alephsub0.com}  \\
  Daniel Lara      &	Matemática	& \url{daniel.lara@alephsub0.com}  \\
  Daniel Lara      &	Matemática	& \url{daniel.lara@alephsub0.com}  \\
  Daniel Lara      &	Matemática	& \url{daniel.lara@alephsub0.com}  \\
  Daniel Lara      &	Matemática	& \url{daniel.lara@alephsub0.com}  \\
  Daniel Lara      &	Matemática	& \url{daniel.lara@alephsub0.com}  \\
  Daniel Lara      &	Matemática	& \url{daniel.lara@alephsub0.com}  \\
  Daniel Lara      &	Matemática	& \url{daniel.lara@alephsub0.com}  \\
  Daniel Lara      &	Matemática	& \url{daniel.lara@alephsub0.com}  \\
  Daniel Lara      &	Matemática	& \url{daniel.lara@alephsub0.com}  \\
  Daniel Lara      &	Matemática	& \url{daniel.lara@alephsub0.com}  \\
  Daniel Lara      &	Matemática	& \url{daniel.lara@alephsub0.com}  \\
  Daniel Lara      &	Matemática	& \url{daniel.lara@alephsub0.com}  \\
  Daniel Lara      &	Matemática	& \url{daniel.lara@alephsub0.com}  \\
  Daniel Lara      &	Matemática	& \url{daniel.lara@alephsub0.com}  \\
  Daniel Lara      &	Matemática	& \url{daniel.lara@alephsub0.com}  \\
  Daniel Lara      &	Matemática	& \url{daniel.lara@alephsub0.com}  \\
  Daniel Lara      &	Matemática	& \url{daniel.lara@alephsub0.com}  \\
  Daniel Lara      &	Matemática	& \url{daniel.lara@alephsub0.com}  \\
  Daniel Lara      &	Matemática	& \url{daniel.lara@alephsub0.com}  \\
  Daniel Lara      &	Matemática	& \url{daniel.lara@alephsub0.com}  \\
  Daniel Lara      &	Matemática	& \url{daniel.lara@alephsub0.com}  \\
  Daniel Lara      &	Matemática	& \url{daniel.lara@alephsub0.com}  \\
  Daniel Lara      &	Matemática	& \url{daniel.lara@alephsub0.com}  \\
  Daniel Lara      &	Matemática	& \url{daniel.lara@alephsub0.com}  \\
  Daniel Lara      &	Matemática	& \url{daniel.lara@alephsub0.com}  \\
  Daniel Lara      &	Matemática	& \url{daniel.lara@alephsub0.com}  \\
  Daniel Lara      &	Matemática	& \url{daniel.lara@alephsub0.com}  \\
  Daniel Lara      &	Matemática	& \url{daniel.lara@alephsub0.com}  \\
  Daniel Lara      &	Matemática	& \url{daniel.lara@alephsub0.com}  \\
  Daniel Lara      &	Matemática	& \url{daniel.lara@alephsub0.com}  \\
  Daniel Lara      &	Matemática	& \url{daniel.lara@alephsub0.com}  \\
 \end{longtable}
 \end{center}

% \begin{lstlisting}[frame=single]
% \begin{tabular}{|c|c|c|}
% \hline
% \textbf{Ciclo N$^\circ$} &	\textbf{Temperatura} & \textbf{Tiempo} \\ \hline
% 1   & 245 5,5 &	3 \\ \hline
% 2	& 260 5,5 &	8 \\ \hline
% \end{tabular}
% \end{lstlisting}

% \begin{center}
%   \begin{tabular}{|c|c|c|}
%   \hline
%     \textbf{Ciclo N$^\circ$} &	\textbf{Temperatura} & \textbf{Tiempo} \\ \hline
%     1   & 245 5,5 &	3 \\ \hline
%     2	& 260 5,5 &	8 \\ \hline
%   \end{tabular}
% \end{center}

\vspace{12pt}

\begin{lstlisting}[frame=single]
\begin{tabular}{|l*{6}{|c}|r|}
    \hline
    Team              & P & W & D & L & F  & A & Pts \\
    \hline
    Manchester United & 6 & 4 & 0 & 2 & 10 & 5 & 12  \\
    Celtic            & 6 & 3 & 0 & 3 &  8 & 9 &  9  \\\hline
\end{tabular}
\end{lstlisting}

\begin{center}
    \begin{tabular}{|l*{6}{|c}|r|}
\hline
Team              & P & W & D & L & F  & A & Pts \\
\hline
Manchester United & 6 & 4 & 0 & 2 & 10 & 5 & 12  \\
Celtic            & 6 & 3 & 0 & 3 &  8 & 9 &  9  \\\hline
\end{tabular}
\end{center}

\vspace{18pt}

\begin{lstlisting}[frame=single]
\begin{tabular}{|c|c|c|}
    \hline
     {medidas}{datos} & 0 & 1\\\hline
    0 & 1 & 2 \\\hline
    0 & 1 & 2 \\\hline
    0 & 1 & 2 \\\hline
\end{tabular}
\end{lstlisting}

\begin{center}
    \begin{tabular}{|c|c|c|}
    \hline
     {medidas}{datos} & 0 & 1\\\hline
    0 & 1 & 2 \\\hline
    0 & 1 & 2 \\\hline
    0 & 1 & 2 \\\hline
\end{tabular}
\end{center}

\end{document} 


