\documentclass[a4,10pt]{aleph-notas}

%% --> Paquetes adicionales
\usepackage{enumitem}
\usepackage{textcomp}
\usepackage{esvect}
\usepackage{amssymb}

%% --> Paquetes comunes
\usepackage{listings}
\usepackage{enumitem}
\usepackage{lipsum}
\usepackage{booktabs}
\usepackage{todonotes}
\usepackage[spanish,onelanguage,vlined,linesnumbered]{algorithm2e}
\setuptodonotes{color=colordef!40, size=\footnotesize}
\newcommand{\porhacer}[1]{\todo[inline]{\textbf{Por hacer:} #1}}

%% --> Definición de colores
\definecolor{codegreen}{HTML}{A5BE00}
\definecolor{codegray}{rgb}{0.5,0.5,0.5}
\definecolor{codepurple}{rgb}{0.58,0,0.82}
\definecolor{backcolour}{rgb}{0.95,0.95,0.92}

%% --> Estilo para código
\lstdefinestyle{mystyle}{
    language={[LaTeX]TeX}, % lenguaje
    basicstyle=\bfseries\ttfamily,
    keywordstyle=\color{colordef},
    commentstyle=\color{codegreen},
    inputencoding=utf8,
    showstringspaces=false,
    flexiblecolumns=true,
    stringstyle=\ttfamily\color{blue},
    extendedchars=true,
    emph={rm,bf,it,sf}, %...
    literate=%
    {ó}{{\'o}}1%
    {í}{{\'i}}1%
    {á}{{\'a}}1%
    {ú}{{\'u}}1%
}

%% --> Selección de estilo para el código
\lstset{
    style=mystyle,escapeinside={(*@}{@*)}
}

% Blancos tipográficos
\newcommand{\mq}{\hspace{0.5em}}  %medio cuadratín
\newcommand{\tq}{\hspace{0.33em}} % un terio de cuadratín
\newcommand{\qq}{\hspace{0.25em}} % un cuarto de cuadratín
\newcommand{\fs}{\hspace{0.125em}} % un octavo de cuadratín
\newcommand{\ep}{\hspace{0.05em}} % espacio de pelo

%% --> Nota para el material
\newcommand{\informacion}{\noindent{\small\color{colordef}
El presente material fue desarrollado por:

\begin{center}
\textbf{Daniel Lara}\\
\emph{Facultad de Ciencias, Escuela Politécnica Nacional}\\[2mm]


\textbf{Andrés Merino}\\
\emph{Facultad de Ciencias Exactas y Naturales, Pontificia Universidad Católica del Ecuador}
\end{center}

\medskip\noindent
La versión actual del material es 1.3-(Noviembre 2021). En caso de encontrar inconsistencias o errores en el presente material se pueden comunicar a \href{mailto:daniel.lara@alephsub0.org}{daniel.lara@alephsub0.org}. Para más información puedes visitar nuestro sitio web: \href{https://alephsub0.org}{alephsub0.org}. Si deseas colaborar con el desarrollo de este material, el código fuente está disponible en:   
\url{https://github.com/alephsub0/LaTeX_Guias.git}. Cualquier aporte (\emph{Pull request}) será de gran ayuda para mejorar este material. 

\medskip\noindent
\includegraphics[height=10pt]{Imagenes/CreativeCommos/cc.xlarge.png}
\includegraphics[height=10pt]{Imagenes/CreativeCommos/by.xlarge.png}
\includegraphics[height=10pt]{Imagenes/CreativeCommos/nc.xlarge.png}
Esta obra se encuentra bajo licencia Atribución-NoComercial-CompartirIgual 4.0 Internacional (CC BY-NC-SA 4.0) Para más información puede visitar: \url{https://creativecommons.org/licenses/by-nc-sa/4.0/}


%% -- > Aquí se incluyen los nombres de los colaboradores de estas guías:
% \medskip\noindent
% Otros colaboradores: Katheryn Yánes
}}

%%--> Formato para títulos
\titleformat{name=\section,numberless}[display]
  {\vspace*{-2mm}\bfseries\scshape\centering}
    {}{1ex}
    {\color{colortext}\large\titlerule\vspace{.05ex}
     }
    [\color{colortext}\vspace{.2ex}\titlerule]

\titleformat{\subsubsection}
    {\color{colortext}\normalsize\bfseries}
    {\thesubsubsection}{1em}{}
    
%% --> Datos de las guias
\universidad{Curso de \LaTeX}
\autor{Proyecto Alephsub0}
\materia{Introducción a \LaTeX}

%% --> Logos de las guias
\logouno[4.5cm]{Imagenes/Logos/LogoAlephsub0-02.png}
\longtitulo{0.6\linewidth}
\fecha{Noviembre de 2021}

%% --> Nuevos ambientes
% \definecolor{coloryt}{rgb}{0.769,0.188,0.169}
%% Ambientes
\makeatletter
%%  Keys temporales: |colorlat|
\def\tcb@@colorlat{colordef!50!black}
    \tcbset{ colorlat/.code = {\def\tcb@@colorlat{#1} } }
%%  Estilo de YouTube
\tcbset{ postitbeta/.style ={
    % -> Opciones generales
    breakable,enhanced,
    before skip=2mm,after skip=3mm,
    colback=\tcb@@color!50,colframe=\tcb@@color!20!black,
    boxrule=0.4pt,
    drop fuzzy shadow,
    left=6mm,right=2mm,top=0.5mm,bottom=0.5mm,
    sharp corners,rounded corners=southeast,arc is angular,arc=3mm,
    parbox=false,
    underlay unbroken and last = {%
        \path[fill=tcbcolback!80!black]
        ([yshift=3mm]interior.south east) --++ (-0.4,-0.1) --++ (0.1,-0.2);
        \path[draw=tcbcolframe,shorten <=-0.05mm,shorten >=-0.05mm]
        ([yshift=3mm]interior.south east) --++ (-0.4,-0.1) --++ (0.1,-0.2);
        \path[fill=\tcb@@colorlat,draw=none]
        (interior.south west) rectangle node[white]{\tcb@@icono} ([xshift=5.5mm]interior.north west);
        },
    underlay = {%
        \path[fill=\tcb@@colorlat,draw=none]
        (interior.south west) rectangle node[white]{\tcb@@icono} ([xshift=5.5mm]interior.north west);
        }
    }
    }
\makeatother

%% Recuadro para enlaces de YouTube
\definecolor{coloryt}{HTML}{ffcccc}
\newtcolorbox{tcbyoutube}
    {icono=\faYoutubePlay,color=coloryt,colorlat=red,postitbeta,colframe=red,leftright skip=1cm}
    
%% Comando para enlaces de YouTube
\newcommand{\YouTube}[4]%
    {
        \begin{tcbyoutube}
            \parbox{0.30\linewidth}{\href{#2}{\includegraphics[width=\linewidth]{#3}}}
            \hspace{2mm}
            \parbox{0.65\linewidth}{\footnotesize
            \textbf{#1}\\[1mm]
            \faLink\ \url{#2}\\[2mm]
            \scriptsize
            #4}
        \end{tcbyoutube}
    }
    
%% Recuadro para enlaces
\definecolor{coloren}{HTML}{B9F2BC}
\newtcolorbox{tcbenlace}
    {icono=\faLink,color=coloren,postit,leftright skip=1cm,fontupper=\small}

%% Recuadro para impresión
\definecolor{colorimp}{HTML}{F8F8FF}
\newtcolorbox{tcbimprimir}
    {icono=\faPrint,color=colorimp,postit,leftright skip=1cm,fontupper=\small}

%% Ambiente para código
\definecolor{colcod}{RGB}{174,218,255}
\newtcolorbox{tcbcodigo}
    {icono=\faCode,color=colcod,postit,top=-2mm,bottom=-2mm,leftright skip=1cm,fontupper=\small}

%% Ambiente para código LaTeX  
\usepackage{minted}
\usemintedstyle{borland}
\tcbuselibrary{minted}
\tcbset{listing engine=minted}
\newtcblisting{tcbLaTeX}{%
    icono=\faCode,color=colcod,postit,top=0mm,bottom=0mm,
    leftright skip=1cm,fontupper=\small,
    minted language=latex,minted style=colorful,
    listing only}
% \newtcolorbox{tcbcodigo}
%     {icono=\faCode,color=colcod,postit,top=-2mm,bottom=-2mm,leftright skip=1cm,fontupper=\small}

%% Ambiente para figuras 
\newtcolorbox[blend into=figures]{figura}[2][]
    {float=h,capture=hbox,title={#2},every float=\centering,
    arc=0mm,left=2mm,right=2mm,
    boxrule=0pt,
    colback=colordef!10,
    colbacktitle=colordef!80,fonttitle=\small,
    enhanced,attach boxed title to bottom,center title,
    #1}

%% Ambiente para tablas 
\newtcolorbox[blend into=tables]{tabla}[2][]
    {float=h,capture=hbox,title={#2},every float=\centering,
    arc=0mm,left=2mm,right=2mm,
    boxrule=0pt,
    colback=colordef!10,
    colbacktitle=colordef!80,fonttitle=\small,
    enhanced,center title,
    #1}

%% Ambiente para código LaTeX desplegado
\newtcblisting{tcbLaTeXb}{%
    icono=\faCode,color=colcod,postit,top=0mm,bottom=0mm,
    fontupper=\small,
    minted language=latex,minted style=colorful,listing side text}
\newtcblisting{tcbLaTeXs}{%
    icono=\faCode,color=colcod,postit,top=0mm,bottom=0mm,
    fontupper=\small,
    minted language=latex,minted style=colorful}

%% Recuadro para comando en línea
\DeclareTotalTCBox{\miverb}{ v }{
    fontupper=\ttfamily,nobeforeafter,tcbox raise base,arc=0pt,outer arc=0pt,
    top=0pt,bottom=0pt,left=0mm,right=0mm,
    leftrule=0pt,rightrule=0pt,toprule=0.3mm,bottomrule=0.3mm,boxsep=0.5mm,
    colback=colcod!10!white,colframe=colcod!50!black}{#1}


% -- Datos del libro
\nota{Guía 3}
\tema{Formato de fórmulas}

%% ---> Opciones adicionales
\newcommand{\Z}{\mathbb{Z}}
\newcommand{\N}{\mathbb{N}}

%% ---> Definiciones de ambientes
\theoremstyle{definition}
\newtheorem{defin}{Definición}

\theoremstyle{plain}
\newtheorem{teor}{Teorema}

%%%%%%%%%%%%%%%%%%%%%%%%%%%%%%%%%%%%%%%%
%%%%%%%%%% Comienzo del documento
%%%%%%%%%%%%%%%%%%%%%%%%%%%%%%%%%%%%%%%%

\begin{document}

\encabezado

\informacion

\color{black}
\normalsize

\tableofcontents

\section{Preliminares}

Para el desarrollo de esta sección vamos a emplear algunos paquetes que complementan las opciones disponibles para la escritura de textos matemáticos. De esta manera, este será el preámbulo necesario para esta guía:

\begin{lstlisting}[frame=single]
\documentclass[10pt]{article}

\usepackage{geometry}

%Paquetes adicionales de simbolos matematicos
\usepackage{amsmath,amssymb,amsfonts,latexsym}
\usepackage[spanish]{babel}
\usepackage[utf8]{inputenc}
\usepackage[T1]{fontenc}
\end{lstlisting}

\begin{advertencia}
En algunos puntos puede ser necesario incluir más paquetes para ciertas actividades; sin embargo, recalcamos la importancia de cargar la menor cantidad de paquetes posibles para evitar posibles inconvenientes en la compilación del documento.
\end{advertencia}

\section{Modo en línea y desplegado}

Para la escritura de texto matemático existen dos formas de acuerdo a si lo queremos dentro de un texto, que se conoce como modo \textit{in line} o si lo queremos desplegado. Así, consideremos el siguiente ejemplo de texto matemático en un párrafo.

\begin{center}
{ \fboxsep 12pt
\fcolorbox {black}{white}{
\begin{minipage}[t]{10cm}
Sea $f:[a,b]\rightarrow \mathbb{R}$ una función. Si $f$ es acotada y discontinua en un número finito de puntos, entonces $f$ es integrable.
\end{minipage}
} }
\end{center}

\noindent
el código del ejemplo anterior es: 

\begin{lstlisting}[frame=single]
Sea $f:[a,b]\rightarrow \mathbb{R}$ una funcion. Si $f$ es acotada y discontinua
en un numero finito de puntos, entonces $f$ es integrable.
\end{lstlisting}

Así, para el modo en linea solo es necesario escribir el texto entre \verb"$... $" o \verb"\(...\)". Por otra parte, para el modo desplegado, consideremos el siguiente ejemplo:

\begin{center}
{ \fboxsep 12pt
\fcolorbox {black}{white}{
\begin{minipage}[t]{10cm}
Sean $f:[a,b]\rightarrow \mathbb{R}$ una función integrable en $[a,b]$ y $c\in[a,b]$. Se tiene que $f$ es integrable en $[a,c]$ y en $[c,b]$, además
    \[
        \int_a^b f(x)\,dx = \int_a^c f(x)\,dx + \int_c^b f(x)\,dx. 
    \]
\end{minipage}
} }
\end{center}

así, el código del ejemplo anterior es:

\begin{lstlisting}[frame=single]
Sean $f:[a,b]\rightarrow \mathbb{R}$ una funcion integrable en $[a,b]$ y $c\in[a,b]$.
Se tiene que $f$ es integrable en $[a,c]$ y en $[c,b]$, ademas
    \[
        \int_a^b f(x)\,dx = \int_a^c f(x)\,dx + \int_c^b f(x)\,dx. 
    \]
\end{lstlisting}

Así, para el modo desplegado escribiremos el texto entre \verb"\[...\]" donde el texto escrito se mostrará centrado y en una nueva linea.

\subsection{Tamaño natural}

Como hemos visto en esta sección \LaTeX{} ajusta el tamaño de las fórmulas de acuerdo a si estas se muestran en una linea o en formato desplegado. Sin embargo, en ocasiones necesitamos mostrar <<a tamaño completo>> una fórmula que se encuentra dentro de una linea. En este caso, vamos a utilizar la siguiente sintaxis:

\begin{lstlisting}[frame=single]
Esta es la formula de una sumatoria $\displaystyle\sum_{k=1}^n n = \frac{n(n+1)}{2}$
\end{lstlisting}

\noindent
cuyo resultado es:

\begin{center}
{ \fboxsep 12pt
\fcolorbox {black}{white}{
\begin{minipage}[t]{10cm}
Esta es la fórmula de una sumatoria $\displaystyle\sum_{k=1}^{n} n = \frac{n(n+1)}{2}$
\end{minipage}
} }
\end{center}


\section{Nuevos comandos}

Desarrollar un nuevo documento requiere de la escritura de comandos o instrucciones que se repiten a lo largo del documento  y que, pasadas algunas lineas, resulta ineficiente. Para ello, vamos a ver algunos ejemplos de «abreviaciones» personalizadas que hemos creado para facilitar la composición de un texto.

\begin{lstlisting}[frame=single]
\newcommand{\sen}{\mathop{\rm sen}}
\newcommand{\arcsen}{\mathop{\rm arcsen}}
\newcommand{\arcsec}{\mathop{\rm arcsec}}
\newcommand{\R}{\mathbb{R}}
\newcommand{\N}{\mathbb{N}}
\newcommand{\Z}{\mathbb{Z}}
\end{lstlisting}

Desarrollar abreviaturas requiere cierta experiencia e ingenio pues se debe hallar un nombre abreviado y que aún no haya sido usado por \LaTeX{} para generar el nuevo comando. En el caso de que se desee reemplazar el contenido de un comando ya existente se reemplaza la instrucción \verb@\newcommand{}@ por \verb@\renewcommand{}@.

\subsection{Comandos con <<parámetros>>}

Un paso más allá de la inclusión de nuevos comandos es el agregar <<parámetros>> a los mismos. Para incluir estos vamos a seguir la siguiente estructura

\begin{lstlisting}[frame=single]
\newcommand{nombre del comando}[numero de parametros]{comando original #1}
\end{lstlisting}

Notemos que la inclusión de los parámetros en el comando está dada por \verb@#1@. Así, el parámetro $1$ va a remplazar la posición \verb@#1@. 

\begin{lstlisting}[frame=single]
\newcommand{\suc}[2][n]{\left(#2\right)_{#1\in\mathbb{\N}}}
\end{lstlisting}

\newcommand{\suc}[2][n]{\left(#2\right)_{#1\in\mathbb{\N}}}

\begin{center}
    \fbox{$\suc{x_n}$}
\end{center}

Para la escritura matemática en general, el proyecto Alephsub0 ha desarrollado un paquete que recopila los comandos matemáticos más usuales además de ciertos ambientes muy útiles a la hora de escribir. Para más información puedes visitar: \href{https://alephsub0.org/recursos-latex/}{https://alephsub0.org/recursos-latex/}. Allí encontrarás el paquete \verb@aleph-comandos.sty@ junto con su documentación. El repositorio con los enlaces de los paquetes para su descarga o vínculo a \emph{Overleaf} los puedes encontrar en: \href{https://git.alephsub0.org/}{https://git.alephsub0.org/}.

%%%%%%%%%%%%%%%%%%%%%%%%%%%%%%%%%%%%%%%%%%%%%%%%%%%%%%%
\section{Subíndices, superíndices}

En la siguiente tabla se encuentran algunos ejemplos sobre el uso de índices y superíndices en la escritura de fórmulas.

\begin{center}
    \begin{tabular}{|c|c|}
        \hline
        Expresión & Código  \\ \hline
        $x^p$ & \verb"x^p" \\ \hline
        $(x_n)_{n\in\mathbb{N}}$ & \verb"(x_n)_{n\in\mathbb{N}}" \\ \hline
        $a_i^j$ & \verb"a_i^j" \\ \hline
        $\int_a^b f(x)\, dx$ & \verb"\int_a^b f(x)\, dx"\\ \hline
        $\sum_{i=1}^n i^2$ & \verb"\sum_{i=1}^n i^2" \\ \hline
        $2^{2^n}$ & \verb"2^{2^n}" \\ \hline 
    \end{tabular}
\end{center}

\section{Fracciones}

En la siguiente tabla se encuentran algunos ejemplos sobre el uso de fracciones con diferentes comandos en la escitura de fórmulas matemáticas:

\begin{center}
    \begin{tabular}{|c|c|}
        \hline
        Expresión & Código  \\ \hline
        $\frac{x+1}{x-1}$ & \verb"\frac{x+1}{x-1}" \\ \hline
        ${x+1 \over; x-1}$ & \verb"{x+1 \over; x-1}" \\ \hline
        $\left( 1+ \frac{1}{x} \right)^\frac{n+1}{n}$ & \verb"\left( 1+ \frac{1}{x} \right)^\frac{n+1}{n}" \\ \hline
        ${x+1 \brace x-1}$ & \verb"{x+1 \brace x-1}"\\ \hline
        ${x+1 \brack x-1}$ & \verb"{x+1 \brack x-1}" \\ \hline
        $a \stackrel{f}{\rightarrow} b$ & \verb"a \stackrel{f}{\rightarrow} b" \\ \hline 
        $\displaystyle{\sum_{\substack{0<i< m\\0<j<n}}a_ib_j}$ & \verb"\displaystyle{\sum_{\substack{0<i< m\\0<j<n}}a_ib_j}" \\ \hline 
        $\prod_{\overset{i=0}{i\neq k}}^{n}\frac{w_i}{(w_i-w_k)}$ & \verb"\prod_{\overset{i=0}{i\neq k}}^{n}\frac{w_i}{(w_i-w_k)}"  \\ \hline
    \end{tabular}
\end{center}

\section{Integrales}

\begin{center}
    \begin{tabular}{|c|c|}
        \hline
        Expresión & Código  \\ \hline
        $\displaystyle{\int_C\boldsymbol{F}\cdot\, dr}$ & \verb"\displaystyle{\int_C\boldsymbol{F}\cdot\, dr}" \\ \hline
        $\displaystyle{\oint_C\pmb{F}\cdot\, dr}$ & \verb"\displaystyle{\oint_C\pmb{F}\cdot\, dr}" \\ \hline
        $\displaystyle{{\iint_D f(x,y)\,dA}}$ & \verb"\displaystyle{{\iint_D f(x,y)\,dA}}" \\ \hline
        $\displaystyle{{\iiint_Q f(x,y,z)\,dA}}$ & \verb"\displaystyle{{\iiint_Q f(x,y,z)\,dA}}"\\ \hline
        $\displaystyle{\iiint\limits_Q}$ & \verb"$\displaystyle{\iiint\limits_Q}" \\ \hline
    \end{tabular}
\end{center}


\section{Raíces}

\begin{center}
    \begin{tabular}{|c|c|}
        \hline
        Expresión & Código  \\ \hline
        $\sqrt{x+1}$ & \verb"\sqrt{x+1}" \\ \hline
        $\displaystyle{ \sqrt[n]{x+\sqrt{x}} }$ & \verb"\displaystyle{ \sqrt[n]{x+\sqrt{x}} }" \\ \hline
    \end{tabular}
\end{center}

\section{Puntos}
    
\begin{center}
\begin{tabular}{|c|c|}
    \hline
    Expresión & Código  \\ \hline
    $\ldots$ & \verb"\ldots" \\ \hline
    $\cdots$ & \verb"\cdots" \\ \hline
    $\vdots$ & \verb"\vdots" \\ \hline
    $\ddots$ & \verb"\ddots" \\ \hline
\end{tabular}
\end{center}

\section{Delimitadores}

\begin{center}
\begin{tabular}{|c|c|}
    \hline
    Expresión & Código  \\ \hline
    $\displaystyle \left[{x+1 \over (x-1)^2} \right]^n$ & \verb"\displaystyle \left[{x+1 \over (x-1)^2} \right]^n" \\ \hline
    $\int_{a}^{b}2x\, dx = \left. x^2 \right|_{a}^{b}$ & \verb"$\int_{a}^{b}2x\, dx = \left. x^2 \right|_{a}^{b}" \\ \hline
\end{tabular}
\end{center}

\subsection{Delimitadores del paquete \texttt{amsmath}}

Una forma alternativa para usar delimitadores que se ajusten a la altura es el uso de los delimitadores proporcionados por el paquete \verb@amsmath@

\bigskip
\noindent
\verb@\Biggl,\Biggr, \biggl,\biggr, \Bigl,\Bigr, \bigl,\bigr@
\bigskip

Así, se tiene el siguiente resultado:

\begin{lstlisting}[frame=single]
\[
    \bigg[ \sum_j \Big| \sum_i x_{ij} \Big|^2 \bigg]^{1/2}
\]
\end{lstlisting}

\[
    \bigg[ \sum_j \Big| \sum_i x_{ij} \Big|^2 \bigg]^{1/2}
\]


\section{Llaves y Barras}

\begin{center}
\begin{tabular}{|c|c|}
    \hline
    Expresión & Código  \\ \hline
    $\overline{AB}$ & \verb"\overline{AB}" \\ \hline
    $\max\left\{a,b\right\}$ & \verb"$\max\left\{a,b\right\}" \\ \hline
    $\overbrace{(x_i-1)}^{K_i}+\underbrace{(x_i-1)}_{K_i}$ & \verb"\overbrace{(x_i-1)}^{K_i}+\underbrace{(x_i-1)}_{K_i}g(x)" \\ \hline
\end{tabular}
\end{center}

\section{Acentos y <<sombreros>>}

Durante la escritura matemática podemos encontrar varios acentos que pueden ser necesarios para ciertas áreas:

\begin{center}
\begin{tabular}{|c|c|}
    \hline
    Expresión & Código  \\ \hline
    $\hat{\imath}$ & \verb"$\hat{\imath}$" \\ \hline
    $\acute{a}$ & \verb"$\acute{a}$" \\ \hline
    $\bar{p}$ & \verb"$\bar{p}$" \\ \hline
    $\vec{p}$ & \verb"$\vec{p}$" \\ \hline
\end{tabular}
\end{center}

\section{Vectores con el paquete \emph{esvect}}

Aunque \LaTeX{} nos provee de una gran cantidad de comandos y herramientas para la escritura de texto matemático, es posible usar paquetes adicionales como es el caso del paquete \verb@esvect@ para presentar una notación más estética dando los siguientes resultados:

\begin{center}
    \begin{tabular}{|c|c|}
        \hline
        Expresión & Código  \\ \hline
        $\vv{v}$ & \verb"$\vv{v}$" \\ \hline
        $\vv{A}$ & \verb"$\vv{A}$" \\ \hline
        $\vv{v \times w}$ & \verb@$\vv{v \times w}$@ \\ \hline
    \end{tabular}
\end{center}

\section{Negritas}

La negrita es un recurso que es necesario incluso en textos matemáticos. A continuación presentamos algunas maneras para colocar negritas en el texto matemático:

\subsection{\texttt{boldsymbol} o \texttt{pmb}}

La sintaxis del comando es muy sencilla y requiere incluir el comando y entre llaves el texto a colocar en negritas.

\begin{lstlisting}[frame=single]
\[
    \pmb{\int_{a}^{b} f(x)\, dx}
\]
\end{lstlisting}

\[
    \pmb{\int_{a}^{b} f(x)\, dx}
\]



\section{Espacios en modo matemático}

\LaTeX{}, por defecto, no deja espacios en modo matemático aunque esto, en general, no representa inconveniente alguno. Sin embargo, si deseamos modificar los espacios podemos usar las siguientes instrucciones:

\begin{itemize}
    \item \verb@\,@ Incluye un espacio igual a $\frac{3}{18}$ de un espacio \verb@\quad@. 
    \item \verb@\:@ Incluye un espacio igual a $\frac{4}{18}$ de un espacio \verb@\quad@. 
    \item \verb@\;@ Incluye un espacio igual a $\frac{5}{18}$ de un espacio \verb@\quad@.
    \item \verb@\@ Incluye un espacio igual a la mitad de un espacio entre palabras.
    \item \verb@\!@ Incluye un espacio igual a $-\frac{3}{18}$ de un espacio \verb@\quad@. 
\end{itemize}

\begin{obs}
El espacio \verb@\quad@ está dado por el ancho de una $M$ en la fuente usada en el documento.
\end{obs}

\section{Arreglos}

\begin{lstlisting}[frame=single]
\[
    A = \left( 
        \begin{array}{lcr}
            a & a+b & k-a \\
            b & b & k-a-b \\
            \vdots & \vdots & \vdots \\
            z & z + z & k-z
        \end{array}
    \right)
\]
\end{lstlisting}

\[
    A = \left( \begin{array}{lcr}
    a & a+b & k-a \\
    b & b & k-a-b \\
    \vdots & \vdots & \vdots \\
    z & z + z & k-z
    \end{array}
    \right)
\]

\begin{lstlisting}[frame=single]
\[
    f(x)= \left\{ 
            \begin{array}{lcl}
                x^2 & \mbox{ si } & x<0 \\
                & & \\
                x-1 & \mbox{ si } & x>0
            \end{array}
        \right.
\]
\end{lstlisting}

\[
    f(x)= \left\{ 
            \begin{array}{lcl}
                x^2 & \mbox{ si } & x<0 \\
                & & \\
                x-1 & \mbox{ si } & x>0
            \end{array}
        \right.
\]

\begin{lstlisting}[frame=single]
\[
    \left\{
        \begin{array}{lclcl}
        \cos x &=& 0 &\Longrightarrow & x=(2k+1)\,\frac{\pi}{2};\; k \in \Z\\
        & & & &\\
        \sen x &=& -1 &\Longrightarrow & x=(4k+3)\,\frac{\pi}{2};\; k \in \Z\\
        & & & &\\
        \cos(2x) &=&\frac{1}{2} &\Longrightarrow &\left\{
            \begin{array}{lcr}
                x &=& \frac{\pi}{6}+k\pi;\; z \in \Z\\
                & &\\
                x &=& -\frac{\pi}{6}+k\pi;\; z \in \Z\\
            \end{array}
        \right.\\
        \end{array}
    \right.
\]
\end{lstlisting}

\[
    \left\{
        \begin{array}{lclcl}
        \cos x &=& 0 &\Longrightarrow & x=(2k+1)\,\frac{\pi}{2};\; k \in \Z\\
        & & & &\\
        \sen x &=& -1 &\Longrightarrow & x=(4k+3)\,\frac{\pi}{2};\; k \in \Z\\
        & & & &\\
        \cos(2x) &=&\frac{1}{2} &\Longrightarrow &\left\{
            \begin{array}{lcr}
                x &=& \frac{\pi}{6}+k\pi;\; z \in \Z\\
                & &\\
                x &=& -\frac{\pi}{6}+k\pi;\; z \in \Z\\
            \end{array}
        \right.\\
        \end{array}
    \right.
\]

\begin{lstlisting}[frame=single]
\[
\begin{array}{rcl}
    A\cup B \subseteq B & & \\ \hline %linea horizontal
    x \in A \cup B &\implies & x \in A \vee x \in B \\[0.15cm]
    %Comentario que recorre tres columnas
    \multicolumn{3}{l}{\mbox{Por hipotesis,\;} x\in A\implies x\in B} \\[0.15cm]
    &\implies & x \in B \vee x \in B \\
    &\implies & x \in B \\
    &\therefore & x A \cup B \subseteq B \\
\end{array}
\]
\end{lstlisting}

\[
    \begin{array}{rcl}
        A\cup B \subseteq B & & \\ \hline %linea horizontal
        x \in A \cup B &\implies & x \in A \vee x \in B \\[0.15cm]
        %Comentario que recorre tres columnas
        \multicolumn{3}{l}{\mbox{Por hipótesis,\;} x\in A\implies x\in B} \\[0.15cm]
        &\implies & x \in B \vee x \in B \\
        &\implies & x \in B \\
        &\therefore & x A \cup B \subseteq B \\
    \end{array}
\]

\section{Matrices}

\begin{lstlisting}[frame=single]
\[
    \begin{pmatrix}
        1 & 0 & 0 &  \cdots & 0 \\
        h_0 & 2(h_0+h_1) & h_1 & \cdots & h_1 \\
        0 & h_1 & 2(h_1+h_2) & 2(h_{n-3}+h_{n-2}) & 0 \\
         \vdots & \vdots & \vdots & \vdots & 9 \\
        0 & h_1 & 2(h_1+h_2) & 2(h_{n-3}+h_{n-2}) & h_{n-2} \\
        6 & 0 & 0 & 0 &  1 \\
    \end{pmatrix}
\]
\end{lstlisting}

\[
    \begin{pmatrix}
        1 & 0 & 0 &  \cdots & 0 \\
        h_0 & 2(h_0+h_1) & h_1 & \cdots & h_1 \\
        0 & h_1 & 2(h_1+h_2) & 2(h_{n-3}+h_{n-2}) & 0 \\
         \vdots & \vdots & \vdots & \vdots & 9 \\
        0 & h_1 & 2(h_1+h_2) & 2(h_{n-3}+h_{n-2}) & h_{n-2} \\
        6 & 0 & 0 & 0 &  1 \\
    \end{pmatrix}
\]

\begin{lstlisting}[frame=single]
\[
    \begin{bmatrix}
        1 & 0 & 0 &  \cdots & 0 \\
        h_0 & 2(h_0+h_1) & h_1 & \cdots & h_1 \\
        0 & h_1 & 2(h_1+h_2) & 2(h_{n-3}+h_{n-2}) & 0 \\
         \vdots & \vdots & \vdots & \vdots & 9 \\
        0 & h_1 & 2(h_1+h_2) & 2(h_{n-3}+h_{n-2}) & h_{n-2} \\
        6 & 0 & 0 & 0 &  1 \\
    \end{bmatrix}
\]
\end{lstlisting}

\[
    \begin{bmatrix}
        1 & 0 & 0 &  \cdots & 0 \\
        h_0 & 2(h_0+h_1) & h_1 & \cdots & h_1 \\
        0 & h_1 & 2(h_1+h_2) & 2(h_{n-3}+h_{n-2}) & 0 \\
         \vdots & \vdots & \vdots & \vdots & 9 \\
        0 & h_1 & 2(h_1+h_2) & 2(h_{n-3}+h_{n-2}) & h_{n-2} \\
        6 & 0 & 0 & 0 &  1 \\
    \end{bmatrix}
\]

\section{Alineamiento}

\begin{lstlisting}[frame=single]
De acuerdo al lema de Euclides tenemos que
\begin{eqnarray*} % Espacio entre filas aumentado \\[0.2cm]
    \mbox{mcd}(a,b) & = & \mbox{mcd}(a-r_0q,r_0) \\[0.2cm]
    & = & \mbox{mcd}(r_1,r_0) \\[0.2cm]
    & = & \mbox{mcd}(r_1,r_0-r_1q_2)\\[0.2cm]
    & = & \mbox{mcd}(r_1,r_2) \\[0.2cm]
    & = & \mbox{mcd}(r_1-r_2q_2,r_2)\\[0.2cm]
\end{eqnarray*}
\end{lstlisting}

De acuerdo al lema de Euclides tenemos que
\begin{eqnarray*} % Espacio entre filas aumentado \\[0.2cm]
\mbox{mcd}(a,b) & = & \mbox{mcd}(a-r_0q,r_0) \\[0.2cm]
& = & \mbox{mcd}(r_1,r_0) \\[0.2cm]
& = & \mbox{mcd}(r_1,r_0-r_1q_2)\\[0.2cm]
& = & \mbox{mcd}(r_1,r_2) \\[0.2cm]
& = & \mbox{mcd}(r_1-r_2q_2,r_2)\\[0.2cm]
\end{eqnarray*}

\section{Ambiente \emph{equation}}

Hasta este punto hemos visto que \LaTeX{} lleva una numeración automática de: capítulos, secciones, subsecciones, entre otros elementos del documento. De la misma manera, el ambiente \verb@equation@ permite llevar una numeración de las <<ecuaciones>> brindando la ventaja de ya ser un modo matemático en si mismo.

\begin{lstlisting}[frame=single]
\begin{equation}
    \int_a^b f(x) \, dx = F(a)-F(b)
\end{equation}
\end{lstlisting}

\begin{equation}
    \int_a^b f(x) \, dx = F(a)-F(b)
\end{equation}

Por otra parte, es posible modificar los contadores de cada una de las <<variables>> empleadas para numerar.

\begin{lstlisting}[frame=single]
\setcounter{equation}{6}
\end{lstlisting}

\setcounter{equation}{6}

\begin{equation}
    S(f,P,C) = \sum_{i,j=1}^n f(c^{ij})\Delta A_{ij}
\end{equation}

Otra opción en el caso de numerar las ecuaciones es usar el ambiente \verb@subequations@ para obtener una subnumeración de las ecuaciones. A continuación se encuentra un ejemplo de su uso:

\begin{lstlisting}[frame=single]
\begin{subequations}
    \begin{equation}
        \log_{2}(xy)=\log_2x + \log_2y
    \end{equation}
    \begin{equation}
        \log_{2}(a^b)=b\log_{2}a
    \end{equation}
\end{subequations}
\end{lstlisting}

\noindent
produce:

\begin{subequations}
    \begin{equation}
        \log_{2}(xy)=\log_2x + \log_2y
    \end{equation}
    \begin{equation}
        \log_{2}(a^b)=b\log_{2}a
    \end{equation}
\end{subequations}

\section{Alineación de ecuaciones}

\subsection{Ambiente \emph{eqnarray}}

Usando el ambiente \verb@eqnarray@ es posible crear arreglos de ecuaciones  de la siguiente forma:

\begin{lstlisting}[frame=single]
\begin{eqnarray} % Espacio entre filas aumentado \\[0.2cm]
    \mbox{mcd}(a,b) & = & \mbox{mcd}(a-r_0q,r_0) \\[0.2cm]
    & = & \mbox{mcd}(r_1,r_0) \\[0.2cm]\nonumber
    & = & \mbox{mcd}(r_1,r_0-r_1q_2)\\[0.2cm]
    & = & \mbox{mcd}(r_1,r_2) \\[0.2cm]
    & = & \mbox{mcd}(r_1-r_2q_2,r_2)\\[0.2cm]
\end{eqnarray}
\end{lstlisting}

\noindent
produce:

\begin{eqnarray} % Espacio entre filas aumentado \\[0.2cm]
\mbox{mcd}(a,b) & = & \mbox{mcd}(a-r_0q,r_0) \\[0.2cm]
& = & \mbox{mcd}(r_1,r_0) \\[0.2cm]\nonumber
& = & \mbox{mcd}(r_1,r_0-r_1q_2)\\[0.2cm]
& = & \mbox{mcd}(r_1,r_2) \\[0.2cm]
& = & \mbox{mcd}(r_1-r_2q_2,r_2)\\[0.2cm]
\end{eqnarray}

\subsection{Ambiente \emph{align}}

\begin{lstlisting}[frame=single]
\begin{align*}
    \intertext{Agrupamos,}
    \frac{a+ay+ax+y}{x+y} &= \frac{ax+ay+x+y}{x+y} &\mbox{Agrupar}\\
    \intertext{sacamos el factor comun,}
    &= \frac{a(x+y)+x+y}{x+y} &\mbox{Factor comun}\\
    &= \frac{(x+y)(a+1)}{x+y} &\mbox{Simplificar}\\
    &= a+1
\end{align*}
\end{lstlisting}

\noindent
produce:

\begin{align*}
\intertext{Agrupamos,}
\frac{a+ay+ax+y}{x+y} &= \frac{ax+ay+x+y}{x+y} &\mbox{Agrupar}\\
\intertext{sacamos el factor común,}
&= \frac{a(x+y)+x+y}{x+y} &\mbox{Factor común}\\
&= \frac{(x+y)(a+1)}{x+y} &\mbox{Simplificar}\\
&= a+1
\end{align*}

\section{Ambientes para escritura de texto matemático}

Los ambientes nos permiten definir estilos de texto y formato para determinadas partes del documento, esto facilita la tarea de modificar más tarde el mismo pues basta cambiar la definición del ambiente y no el contenido del documento como tal. Un ejemplo se puede encontrar a continuación: La primera parte del código, define el ambiente y selecciona su tipo:

\begin{lstlisting}[frame=single]
\theoremstyle{definition}
\newtheorem{defin}{Definicion}

\theoremstyle{plain}
\newtheorem{teor}{Teorema}
\end{lstlisting}

El formato del ambiente está dado por la selección realizada en el comando \verb@\theoremstyle{@\emph{estilo}\verb@}@, los formatos disponibles son:

\begin{itemize}
    \item \verb@plain@ texto en itálica, espacio extra arriba y abajo;
    \item \verb@definition@ texto en romana\footnote{Romana se refiere a la fuente normal o «común» en inglés denominada \emph{upright}.}, espacio extra arriba y abajo;
    \item \verb@remark@ texto en romana, no agrega espacio extra.
\end{itemize}

Así, tenemos el siguiente código de ejemplo:

\begin{lstlisting}[frame=single]
\begin{defin}[Subconjunto]
Sean $A$ y $B$ conjuntos, se dice que $A$ es subconjunto de $B$ (y se nota $A\subseteq B$)
si y solo si
\[
    x\in A\Rightarrow x\in B.
\]
\end{defin}


\begin{teor}
Sean $A$ y $B$ conjuntos, si $A\subseteq B$ y $B\subseteq A$ entonces $A=B$.
\end{teor}
\end{lstlisting}

\noindent
produce:
\begin{defin}[Subconjunto]
Sean $A$ y $B$ conjuntos, se dice que $A$ es subconjunto de $B$ (y se nota $A\subseteq B$) si y solo si
\[
    x\in A\Rightarrow x\in B.
\]
\end{defin}


\begin{teor}
Sean $A$ y $B$ conjuntos, si $A\subseteq B$ y $B\subseteq A$ entonces $A=B$.
\end{teor}

\begin{advertencia}
Note las diferencias entre el formato definido para teorema y definición, ¿concuerdan con lo mencionado respecto a los estilos de teorema?
\end{advertencia}

\section{Ejercicios}

A continuación se encuentran algunas fórmulas y desarrollos matemáticos, sugerimos que el lector desarrolle los mismos a fin de entrenarse en el uso de \LaTeX{}.

\paragraph{Ejercicio 1.} Deducir la identidad $\cos(\alpha)\cos(\beta)=\frac 1 2 {\cos(\alpha+\beta)+\cos(\alpha-\beta)}$.

\begin{proof}[Solución]
    Sean $\alpha,\ \beta\in\mathbb{R}$, se sigue que
    \begin{equation}\label{eq:001}
        \cos(\alpha+\beta)=\cos(\alpha)\cos(\beta)-\sen(\alpha)\sen(\beta),
    \end{equation}
    y
    \begin{equation}\label{eq:002}
        \cos(\alpha-\beta)=\cos(\alpha)\cos(\beta)+\sen(\alpha)\sen(\beta).
    \end{equation}
    Sumando las identidades \eqref{eq:001} y \eqref{eq:002}
    \begin{equation*}
      \cos(\alpha+\beta)+\cos(\alpha-\beta)=2\cos(\alpha)\cos(\beta),
    \end{equation*}
    por lo tanto
    \begin{equation*}
      \cos(\alpha)\cos(\beta)=\frac {\cos(\alpha+\beta)+\cos(\alpha-\beta)} 2.\qedhere
    \end{equation*}
\end{proof}

\paragraph{Ejercicio 2.} Calcular $\cos(u+\frac \pi 2)$.
\begin{proof}[Solución]
Conociendo que $\cos(\frac \pi 2)=0$ y $\sen(\frac \pi 2)=1$, se sigue que
\begin{align*}
 \cos\left(u+\frac \pi 2\right)&=\cos(u)\cos\left(\frac \pi 2\right)-\sen(u)\sen\left(\frac \pi 2\right)\\
    &=0\cos(u)-1\sen(u)\\
    &=-\sen(u)\qedhere
\end{align*}

\end{proof}

\begin{itemize}
\item Desarrollo de formulas:
\begin{align}
L_j(x)&=\prod_{\substack{i=1\\i\not=j}}^n\frac{(x-x_i)}{(x_j-x_i)}\\
  &=\frac{(x-x_1)(x-x_2)\cdots(x-x_{j-1})(x-x_{j+1})\cdots(x-x_n)}{(x_j-x_1)(x_j-x_2)\cdots(x_j-x_{j-1})(x_j-x_{j+1})\cdots(x_j-x_n)}\notag
\end{align}

\item Formulas largas:
\begin{align}
f_h (x,y)&=E\int_0^t L_{x,y_\varphi}du\notag\\
&= h\int L_{x, z} dz \notag\\
&\phantom{=} {} + h \biggl[ \frac 1 {t}\bigg( E \int_0^{t}  L_{x, y^x}ds- t\int L_{x,z}dz) \biggr)  \notag\\
&\phantom{= {} + h \biggl[} +\frac 1 {t}\bigg( E \int_0^{t}  L_{x, y^x} ds - E\int_0^t L_{x,y_\varphi} ds \biggr)\biggr]
\end{align}

\item Funciones
\begin{equation*}
\begin{array}{rcl}
f\colon \mathbb{R}&\longrightarrow&\mathbb{R}\\
x&\longmapsto&f(x)
\end{array}
\end{equation*}

\begin{equation*}
\delta_{i,j}=\begin{cases}
0&\text{si }i=j\\
1&\text{si }i\not=j
\end{cases}
\end{equation*}

\item Sistemas de ecuaciones
\begin{equation}
\left\{\begin{array}{cccccc}
x&+y&+z&+w&=&1\\
&3y&&+5w&=&2\\
5x&&-2z&&=&4
\end{array}\right.
\end{equation}

\item Matrices
\begin{equation*}
\begin{bmatrix}
1\\2\\3\\4\\5
\end{bmatrix}
\end{equation*}

\begin{gather}
A=\begin{pmatrix}
a_{11}&\cdots&a_{1n}\\
\vdots&\ddots&\vdots\\
a_{n1}&\cdots&a_{nn}
\end{pmatrix}\\
\Delta_A(\lambda)=|\lambda I-A|=\begin{vmatrix}
\lambda-a_{11}&\cdots&a_{1n}\\
\vdots&\ddots&\vdots\\
a_{n1}&\cdots&\lambda-a_{nn}
\end{vmatrix}
\end{gather}
\item Signos de agrupación

\begin{equation*}
\left.a\left(\frac{\left(\dfrac{a+t}{a-t}+t!\right)^{t-1}}{\sen^2 t+\sen^2 t}-\sqrt[5]{t+1}\right)\right|_{t=0}
\end{equation*}
\end{itemize}


\section{Referencias}

Para más información detallada sobre el funcionamiento del modo matemático puede consultar los siguientes vínculos:

\begin{itemize}
    \item 
        \url{https://tex.loria.fr/general/Voss-Mathmode.pdf}
    \item
        \url{http://mirrors.ucr.ac.cr/CTAN/info/symbols/comprehensive/}
\end{itemize}

\end{document} 