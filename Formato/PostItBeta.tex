\makeatletter
%%  Keys temporales: |colorlat|
\def\tcb@@colorlat{colordef!50!black}
    \tcbset{ colorlat/.code = {\def\tcb@@colorlat{#1} } }
%%  Estilo de YouTube
\tcbset{ postitbeta/.style ={
    % -> Opciones generales
    breakable,enhanced,
    before skip=2mm,after skip=3mm,
    colback=\tcb@@color!50,colframe=\tcb@@color!20!black,
    boxrule=0.4pt,
    drop fuzzy shadow,
    left=6mm,right=2mm,top=0.5mm,bottom=0.5mm,
    sharp corners,rounded corners=southeast,arc is angular,arc=3mm,
    parbox=false,
    underlay unbroken and last = {%
        \path[fill=tcbcolback!80!black]
        ([yshift=3mm]interior.south east) --++ (-0.4,-0.1) --++ (0.1,-0.2);
        \path[draw=tcbcolframe,shorten <=-0.05mm,shorten >=-0.05mm]
        ([yshift=3mm]interior.south east) --++ (-0.4,-0.1) --++ (0.1,-0.2);
        \path[fill=\tcb@@colorlat,draw=none]
        (interior.south west) rectangle node[white]{\tcb@@icono} ([xshift=5.5mm]interior.north west);
        },
    underlay = {%
        \path[fill=\tcb@@colorlat,draw=none]
        (interior.south west) rectangle node[white]{\tcb@@icono} ([xshift=5.5mm]interior.north west);
        }
    }
    }
\makeatother

%% Recuadro para enlaces de YouTube
\definecolor{coloryt}{HTML}{ffcccc}
\newtcolorbox{tcbyoutube}
    {icono=\faYoutubePlay,color=coloryt,colorlat=red,postitbeta,colframe=red,leftright skip=1cm}
    
%% Comando para enlaces de YouTube
\newcommand{\YouTube}[4]%
    {
        \begin{tcbyoutube}
            \parbox{0.30\linewidth}{\href{#2}{\includegraphics[width=\linewidth]{#3}}}
            \hspace{2mm}
            \parbox{0.65\linewidth}{\footnotesize
            \textbf{#1}\\[1mm]
            \faLink\ \url{#2}\\[2mm]
            \scriptsize
            #4}
        \end{tcbyoutube}
    }
    
%% Recuadro para enlaces
\definecolor{coloren}{HTML}{B9F2BC}
\newtcolorbox{tcbenlace}
    {icono=\faLink,color=coloren,postit,leftright skip=1cm,fontupper=\small}

%% Recuadro para impresión
\definecolor{colorimp}{HTML}{F8F8FF}
\newtcolorbox{tcbimprimir}
    {icono=\faPrint,color=colorimp,postit,leftright skip=1cm,fontupper=\small}

%% Ambiente para código
\definecolor{colcod}{RGB}{174,218,255}
\newtcolorbox{tcbcodigo}
    {icono=\faCode,color=colcod,postit,top=-2mm,bottom=-2mm,leftright skip=1cm,fontupper=\small}

%% Ambiente para código LaTeX  
\usepackage{minted}
\usemintedstyle{borland}
\tcbuselibrary{minted}
\tcbset{listing engine=minted}
\newtcblisting{tcbLaTeX}{%
    icono=\faCode,color=colcod,postit,top=0mm,bottom=0mm,
    leftright skip=1cm,fontupper=\small,
    minted language=latex,minted style=colorful,
    listing only}
% \newtcolorbox{tcbcodigo}
%     {icono=\faCode,color=colcod,postit,top=-2mm,bottom=-2mm,leftright skip=1cm,fontupper=\small}

%% Ambiente para figuras 
\newtcolorbox[blend into=figures]{figura}[2][]
    {float=h,capture=hbox,title={#2},every float=\centering,
    arc=0mm,left=2mm,right=2mm,
    boxrule=0pt,
    colback=colordef!10,
    colbacktitle=colordef!80,fonttitle=\small,
    enhanced,attach boxed title to bottom,center title,
    #1}

%% Ambiente para tablas 
\newtcolorbox[blend into=tables]{tabla}[2][]
    {float=h,capture=hbox,title={#2},every float=\centering,
    arc=0mm,left=2mm,right=2mm,
    boxrule=0pt,
    colback=colordef!10,
    colbacktitle=colordef!80,fonttitle=\small,
    enhanced,center title,
    #1}

%% Ambiente para código LaTeX desplegado
\newtcblisting{tcbLaTeXb}{%
    icono=\faCode,color=colcod,postit,top=0mm,bottom=0mm,
    fontupper=\small,
    minted language=latex,minted style=colorful,listing side text}
\newtcblisting{tcbLaTeXs}{%
    icono=\faCode,color=colcod,postit,top=0mm,bottom=0mm,
    fontupper=\small,
    minted language=latex,minted style=colorful}
